
% mu-benchmarks.tex

\subsection{Micro-benchmarks record observables}

Micro-benchmarks are recording observables that measure resource usage of the
whole program for a specific time. These measurements are then associated with
the subsystem that was observed at that time.
Caveat: if the executable under observation runs on a multiprocessor computer
where more than one parallel thread executes at the same time, it becomes
difficult to associate resource usage to a single function. Even more so, as
Haskell's thread do not map directly to operating system threads. So the
expressiveness of our approach is only valid statistically when a large number
of observables have been captured.

\subsubsection{Requirements}


\subsubsection{Implementing micro-benchmarks}

In a micro-benchmark we capture operating system counters before a function of
interest is run, and afterwards. Then, we compute the difference between the
two and report all three measurements via a |Trace| to the logging system.


\subsubsection{Configuration of mu-benchmarks}

